\documentclass[lang=cn,10pt]{elegantbook}

\title{M2 Trigonometry}
\subtitle{含2012-2025年与Trigonometryn相关的全部题目,排版舒服,方便练习}

\author{Yau YukMing}
\institute{莘睿未来}
\date{Aug 27, 2025}
\version{1.2}
\bioinfo{Level}{Senior}

\extrainfo{No pain, No gain}

\setcounter{tocdepth}{3}

\logo{ShinerayFutureLogo}
\cover{Cover.png}

% 本文档命令
\usepackage{array}
\usepackage{graphicx,subfig}
\newcommand{\ccr}[1]{\makecell{{\color{#1}\rule{1cm}{1cm}}}}
\usepackage{draftwatermark}
\SetWatermarkText{ShineRayFuture}
\SetWatermarkLightness{0.9}             % 设置水印透明度 0-1
\SetWatermarkScale{0.5}
% 修改标题页的橙色带
% \definecolor{customcolor}{RGB}{32,178,170}
% \colorlet{coverlinecolor}{customcolor}

\begin{document}

\maketitle
\frontmatter

\mainmatter
\begin{problem}{2012Q9}
\newline
(a) Using integration by parts, find $\int x \sin x \mathrm{~d} x$.\\
(b)
\begin{figure}[h]
    \centering
    \includegraphics[width=0.5\linewidth]{2012Q9.png}
    \caption{2012Q9}
    \label{fig:placeholder}
\end{figure}
\newline
Figure 4 shows the shaded region bounded by the curve $y=\sqrt{x \sin x}$ for $0 \leq x \leq \pi$ and the $x$-axis. Find the volume of the solid generated by revolving the region about the $x$-axis.
\begin{flushright}
(4 marks)    
\end{flushright}
\end{problem}
\newpage
continue 2012Q9
\newpage
\begin{problem}{2012Q10}
\newline
\begin{figure}[h]
    \centering
    \includegraphics[width=0.5\linewidth]{2012Q10.png}
    \caption{2012Q10}
    \label{fig:placeholder}
\end{figure}
\newline
In Figure 5, $O A B$ is an isosceles triangle with $O A=O B, A B=1, A Y=y, \angle A O Y=\theta$ and $\angle B O Y=3 \theta$.\\
(a) Show that $y=\frac{1}{4} \sec ^2 \theta$.\\
(b) Find the range of values of $y$.\\
Hint: you may use the identity $\sin 3 \theta=3 \sin \theta-4 \sin ^3 \theta$.
\begin{flushright}
(6 marks)    
\end{flushright}
\end{problem}
\newpage
continue 2012Q10
\newpage
\begin{problem}{2012Q13}
\newline
(a) (i) Suppose $\tan u=\frac{-1+\cos \frac{2 \pi}{5}}{\sin \frac{2 \pi}{5}}$, where $\frac{-\pi}{2}<u<\frac{\pi}{2}$.\\

Show that $u=\frac{-\pi}{5}$.\\
(ii) Suppose $\tan \nu=\frac{1+\cos \frac{2 \pi}{5}}{\sin \frac{2 \pi}{5}}$.\\

Find $v$, where $\frac{-\pi}{2}<v<\frac{\pi}{2}$.
\begin{flushright}
(4 marks)    
\end{flushright}
(b) (i) Express $x^2+2 x \cos \frac{2 \pi}{5}+1$ in the form $(x+a)^2+b^2$, where $a$ and $b$ are constants.\\
(ii) Evaluate $\int_{-1}^1 \frac{\sin \frac{2 \pi}{5}}{x^2+2 x \cos \frac{2 \pi}{5}+1} \mathrm{~d} x$.
\begin{flushright}
(6 marks)    
\end{flushright}
(c) Evaluate $\int_{-1}^1 \frac{\sin \frac{7 \pi}{5}}{x^2+2 x \cos \frac{7 \pi}{5}+1} \mathrm{~d} x$.
\begin{flushright}
(3 marks)    
\end{flushright}
\end{problem}
\newpage
continue 2012Q13
\newpage
\begin{problem}{2013Q1}
\newline
Find $\frac{\mathrm{d}}{\mathrm{d} x}(\sin 2 x)$ from first principles.
\begin{flushright}
(4 marks)    
\end{flushright}
\end{problem}
\newpage
\begin{problem}{2013Q7}
\newline
(a) Prove the identity $\tan x=\frac{\sin 2 x}{1+\cos 2 x}$.\\
(b) Using (a), prove the identity $\tan y=\frac{\sin 8 y \cos 4 y \cos 2 y}{(1+\cos 8 y)(1+\cos 4 y)(1+\cos 2 y)}$.
\begin{flushright}
(5 marks)    
\end{flushright}
\end{problem}
\newpage
\begin{problem}{2013Q11}
\newline
(a) Let $0<\theta<\frac{\pi}{2}$. By finding $\frac{\mathrm{d}}{\mathrm{d} \theta} \ln (\sec \theta+\tan \theta)$, or otherwise, show that $\int \sec \theta \mathrm{d} \theta=\ln (\sec \theta+\tan \theta)+C$, where $C$ is any constant.
\begin{flushright}
(2 marks)    
\end{flushright}
(b) (i) Using (a) and a suitable substitution, show that $\int \frac{\mathrm{d} u}{\sqrt{u^2-1}}=\ln \left(u+\sqrt{u^2-1}\right)+C$ for $u>1$.\\
(ii) Using (b)(i), show that $\int_0^1 \frac{2 x}{\sqrt{x^4+4 x^2+3}} \mathrm{~d} x=\ln (6+4 \sqrt{2}-3 \sqrt{3}-2 \sqrt{6})$.
\begin{flushright}
(5 marks)    
\end{flushright}
(c) Let $t=\tan \phi$. Show that $\frac{\mathrm{d} \phi}{\mathrm{d} t}=\frac{1}{1+t^2}$.\\
Hence evaluate $\int_0^{\frac{\pi}{4}} \frac{\tan \phi}{\sqrt{1+2 \cos ^2 \phi}} \mathrm{~d} \phi$.
\begin{flushright}
(5 marks)    
\end{flushright}
\end{problem}
\newpage
continue 2013Q11
\newpage
\begin{problem}{2013Q12}
\newline
\begin{figure}[h]
    \centering
    \includegraphics[width=0.5\linewidth]{2013Q12png}
    \caption{2013Q12}
    \label{fig:placeholder}
\end{figure}
\newline
In Figure 3, the distance between two houses $A$ and $B$ lying on a straight river bank is 40 m . The width of the river is always 30 m . In the beginning, Mike stands at the starting point $P$ in the opposite bank which is 30 m from $A$. Mike's wife, situated at $A$, is watching him running along the bank for $x \mathrm{~m}$ at a constant speed of $7 \mathrm{~m} \mathrm{~s}^{-1}$ to point $Q$ and then swimming at a constant speed of $1.4 \mathrm{~m} \mathrm{~s}^{-1}$ along a straight path to reach $B$.\\
(a) Let $T$ seconds be the time that Mike travels from $P$ to $B$.\\
(i) Express $T$ in terms of $x$.\\
(ii) When $T$ is minimum, show that $x$ satisfies the equation $2 x^2-160 x+3125=0$. Hence show that $Q B=\frac{25 \sqrt{6}}{2} \mathrm{~m}$.
\begin{flushright}
(6 marks)    
\end{flushright}
(b) In Figure 4, Mike is swimming from $Q$ to $B$ with $Q B$ equal to the value mentioned in (a)(ii). Let $\angle M A B=\alpha$ and $\angle A B M=\beta$, where $M$ is the position of Mike.\\
(i) By finding $\sin \beta$ and $\cos \beta$, show that $M B=\frac{200 \tan \alpha}{\tan \alpha+2 \sqrt{6}}$.\\
(ii) Find the rate of change of $\alpha$ when $\alpha=0.2$ radian. Correct your answer to 4 decimal places.
\begin{flushright}
(7 marks)    
\end{flushright}
\end{problem}
\newpage
continue 2013Q12
\newpage
\begin{problem}{2014Q4}
\newline
Let $x=2 y+\sin y$. Find $\frac{\mathrm{d}^2 y}{\mathrm{~d} x^2}$ in terms of $y$.
\begin{flushright}
(3 marks)    
\end{flushright}
\end{problem}
\newpage
\begin{problem}{2014Q13}
\newline
(a) Prove that $1-\cos 4 \theta-2 \cos 2 \theta \sin ^2 2 \theta=16 \cos ^2 \theta \sin ^4 \theta$.
\begin{flushright}
(2 marks)    
\end{flushright}
(b) Show that $\int_0^{n \pi} \cos ^2 x \sin ^4 x \mathrm{~d} x=\frac{n \pi}{16}$, where $n$ is a positive integer.
\begin{flushright}
(4 marks)    
\end{flushright}
\newpage
(c) Let $\mathrm{f}(x)$ be a continuous function such that $\mathrm{f}(k-x)=\mathrm{f}(x)$, where $k$ is a constant. Show that $\int_0^k x \mathrm{f}(x) \mathrm{d} x=\frac{k}{2} \int_0^k \mathrm{f}(x) \mathrm{d} x$.
\begin{flushright}
(4 marks)    
\end{flushright}
(d)
\begin{figure}[h]
    \centering
    \includegraphics[width=0.5\linewidth]{2014Q13.png}
    \caption{2014Q13}
    \label{fig:placeholder}
\end{figure}
\newline
Figure 5 shows the shaded region bounded by the curve $y=\cos ^2 x \sin ^4 \dot{x}$ and the $x$-axis, where $\pi \leq x \leq 2 \pi$. Find the volume of the solid of revolution when the shaded region is revolved about the $y$-axis.

(4 marks)
\end{problem}
\newpage
continue 2014Q13
\newpage
\begin{problem}{2015Q2}
\newline
Let $y=x \sin x+\cos x$.\\
(a) Find $\frac{\mathrm{d} y}{\mathrm{~d} x}$ and $\frac{\mathrm{d}^2 y}{\mathrm{~d} x^2}$.\\
(b) Let $k$ be a constant such that $x \frac{\mathrm{~d}^2 y}{\mathrm{~d} x^2}+k \frac{\mathrm{~d} y}{\mathrm{~d} x}+x y=0$ for all real values of $x$. Find the value of $k$.
\begin{flushright}
(5 marks)    
\end{flushright}
\end{problem}
\newpage
\begin{problem}{2015Q7}
\newline
(a) Prove that $\sin ^2 x \cos ^2 x=\frac{1-\cos 4 x}{8}$.\\
(b) Let $\mathrm{f}(x)=\sin ^4 x+\cos ^4 x$.\\
(i) Express $\mathrm{f}(x)$ in the form $A \cos B x+C$, where $A, B$ and $C$ are constants.\\
(ii) Solve the equation $8 \mathrm{f}(x)=7$, where $0 \leq x \leq \frac{\pi}{2}$.
\begin{flushright}
(7 marks)    
\end{flushright}
\end{problem}
\newpage
\begin{problem}{2015Q8}
\newline
(a) Using mathematical induction, prove that $\sin \frac{x}{2} \sum_{k=1}^n \cos k x=\sin \frac{n x}{2} \cos \frac{(n+1) x}{2}$ for all positive integers $n$.\\
(b) Using (a), evaluate $\sum_{k=1}^{567} \cos \frac{k \pi}{7}$.
\begin{flushright}
(8 marks)    
\end{flushright}
\end{problem}
\newpage
\begin{problem}{2016Q6}
\newline
(a) Prove that $x+1$ is a factor of $4 x^3+2 x^2-3 x-1$.\\
(b) Express $\cos 3 \theta$ in terms of $\cos \theta$.\\
(c) Using the results of (a) and (b), prove that $\cos \frac{3 \pi}{5}=\frac{1-\sqrt{5}}{4}$.
\begin{flushright}
(6 marks)    
\end{flushright}
\end{problem}
\newpage
\begin{problem}{2016Q10}
\newline
(a) Let $\mathrm{f}(x)$ be a continuous function defined on the interval $[0, a]$, where $a$ is a positive constant. Prove that $\int_0^a f(x) d x=\int_0^a f(a-x) d x$.
\begin{flushright}
(3 marks)    
\end{flushright}
(b) Prove that $\int_0^{\frac{\pi}{4}} \ln (1+\tan x) \mathrm{d} x=\int_0^{\frac{\pi}{4}} \ln \left(\frac{2}{1+\tan x}\right) \mathrm{d} x$.
\begin{flushright}
(3 marks)    
\end{flushright}
(c) Using (b), prove that $\int_0^{\frac{\pi}{4}} \ln (1+\tan x) \mathrm{d} x=\frac{\pi \ln 2}{8}$.
\begin{flushright}
(3 marks)    
\end{flushright}
(d) Using integration by parts, evaluate $\int_0^{\frac{\pi}{4}} \frac{x \sec ^2 x}{1+\tan x} \mathrm{~d} x$.
\begin{flushright}
(3 marks)    
\end{flushright}
\end{problem}
\newpage
continue 2016Q10
\newpage
\begin{problem}{2017Q1}
\newline
Find $\frac{\mathrm{d}}{\mathrm{d} \theta} \sec 6 \theta$ from first principles.
\begin{flushright}
(5 marks)    
\end{flushright}

\end{problem}
\newpage
\begin{problem}{2017Q7}
\newline
(a) Prove that $\sin 3 x=3 \sin x-4 \sin ^3 x$.\\
(b) Let $\frac{\pi}{4}<x<\frac{\pi}{2}$.\\
(i) Prove that $\frac{\sin 3\left(x-\frac{\pi}{4}\right)}{\sin \left(x-\frac{\pi}{4}\right)}=\frac{\cos 3 x+\sin 3 x}{\cos x-\sin x}$.\\
(ii) Solve the equation $\frac{\cos 3 x+\sin 3 x}{\cos x-\sin x}=2$.
\begin{flushright}
(8 marks)    
\end{flushright}
\end{problem}
\newpage
\begin{problem}{2017Q11}
\newline
(a) Using $\tan ^{-1} \sqrt{2}-\tan ^{-1}\left(\frac{\sqrt{2}}{2}\right)=\tan ^{-1}\left(\frac{\sqrt{2}}{4}\right)$, evaluate $\int_0^1 \frac{1}{x^2+2 x+3} \mathrm{~d} x$.
\begin{flushright}
(3 marks)    
\end{flushright}
(b) (i) Let $0 \leq \theta \leq \frac{\pi}{4}$. Prove that $\frac{2 \tan \theta}{1+\tan ^2 \theta}=\sin 2 \theta$ and $\frac{1-\tan ^2 \theta}{1+\tan ^2 \theta}=\cos 2 \theta$.
(ii) Using the substitution $t=\tan \theta$, evaluate $\int_0^{\frac{\pi}{4}} \frac{1}{\sin 2 \theta+\cos 2 \theta+2} \mathrm{~d} \theta$.
\begin{flushright}
(5 marks)    
\end{flushright}
(c) Prove that $\int_0^{\frac{\pi}{4}} \frac{\sin 2 \theta+1}{\sin 2 \theta+\cos 2 \theta+2} d \theta=\int_0^{\frac{\pi}{4}} \frac{\cos 2 \theta+1}{\sin 2 \theta+\cos 2 \theta+2} d \theta$.
\begin{flushright}
(2 marks)    
\end{flushright}
(d) Evaluate $\int_0^{\frac{\pi}{4}} \frac{8 \sin 2 \theta+9}{\sin 2 \theta+\cos 2 \theta+2} \mathrm{~d} \theta$.
\begin{flushright}
(3 marks)    
\end{flushright}
\end{problem}
\newpage
continue 2017Q11
\newpage
\begin{problem}{2018Q3}
\newline
(a) If $\cot A=3 \cot B$, prove that $\sin (A+B)=2 \sin (B-A)$.\\
(b) Using (a), solve the equation $\cot \left(x+\frac{4 \pi}{9}\right)=3 \cot \left(x+\frac{5 \pi}{18}\right)$, where $0 \leq x \leq \frac{\pi}{2}$.
\begin{flushright}
(5 marks)    
\end{flushright}
\end{problem}
\newpage
\begin{problem}{2018Q10}
\newline
(a) (i) Prove that $\int \sin ^4 x \mathrm{~d} x=\frac{-\cos x \sin ^3 x}{4}+\frac{3}{4} \int \sin ^2 x \mathrm{~d} x$.
(ii) Evaluate $\int_0^\pi \sin ^4 x \mathrm{~d} x$.
\begin{flushright}
(5 marks)    
\end{flushright}
(b) (i) Let $\mathrm{f}(x)$ be a continuous function such that $\mathrm{f}(\beta-x)=\mathrm{f}(x)$ for all real numbers $x$, where $\beta$ is a constant. Prove that $\int_0^\beta x \mathrm{f}(x) \mathrm{d} x=\frac{\beta}{2} \int_0^\beta \mathrm{f}(x) \mathrm{d} x$.\\
(ii) Evaluate $\int_0^\pi x \sin ^4 x \mathrm{~d} x$.
\begin{flushright}
(5 marks)    
\end{flushright}
(c) Consider the curve $G: y=\sqrt{x} \sin ^2 x$, where $\pi \leq x \leq 2 \pi$. Let $R$ be the region bounded by $G$ and the $x$-axis. Find the volume of the solid of revolution generated by revolving $R$ about the $x$-axis.
\begin{flushright}
(3 marks)    
\end{flushright}
\end{problem}
\newpage
continue 2018Q10
\newpage
\begin{problem}{2019Q7}
\newline
(a) Using integration by parts, find $\int e^x \sin \pi x \mathrm{~d} x$.\\
(b) Using integration by substitution, evaluate $\int_0^3 e^{3-x} \sin \pi x \mathrm{~d} x$.
\begin{flushright}
(7 marks)    
\end{flushright}
\end{problem}
\newpage
\begin{problem}{2019Q10}
\newline
(a) Let $0 \leq x \leq \frac{\pi}{4}$. Prove that $\frac{1}{2+\cos 2 x}=\frac{\sec ^2 x}{2+\sec ^2 x}$.
\begin{flushright}
(1 mark)    
\end{flushright}
(b) Evaluate $\int_0^{\frac{\pi}{4}} \frac{1}{2+\cos 2 x} \mathrm{~d} x$.
\begin{flushright}
(3 marks)    
\end{flushright}
(c) Let $\mathrm{f}(x)$ be a continuous function defined on $\mathbf{R}$ such that $\mathrm{f}(-x)=-\mathrm{f}(x)$ for all $x \in \mathbf{R}$.\\

Prove that $\int_{-a}^a \mathrm{f}(x) \ln \left(1+e^x\right) \mathrm{d} x=\int_0^a x \mathrm{f}(x) \mathrm{d} x$ for any $a \in \mathbb{R}$.
\begin{flushright}
(4 marks)    
\end{flushright}
(d) Evaluate $\int_{\frac{-\pi}{4}}^{\frac{\pi}{4}} \frac{\sin 2 x}{(2+\cos 2 x)^2} \ln \left(1+e^x\right) \mathrm{d} x$.
\begin{flushright}
(5 marks)    
\end{flushright}
\end{problem}
\newpage
continue 2019Q10
\newpage
\begin{problem}{2020Q3}
\newline
(a) Let $x$ be an angle which is not a multiple of $30^{\circ}$. Prove that\\
(i) $\quad \tan 3 x=\frac{3 \tan x-\tan ^3 x}{1-3 \tan ^2 x}$,\\
(ii) $\quad \tan x \tan \left(60^{\circ}-x\right) \tan \left(60^{\circ}+x\right)=\tan 3 x$.\\
(b) Using (a)(ii), prove that $\tan 55^{\circ} \tan 65^{\circ} \tan 75^{\circ}=\tan 85^{\circ}$.
\begin{flushright}
(6 marks)    
\end{flushright}
\end{problem}
\newpage
\begin{problem}{2020Q4}
\newline
(a) Find $\int \sin ^2 \theta \mathrm{~d} \theta$.\\
(b) Define $\mathrm{f}(x)=4 x\left(1-x^2\right)^{\frac{1}{4}}$ for all $x \in[0,1]$. Denote the graph of $y=\mathrm{f}(x)$ by $G$. Let $R$ be the region bounded by $G$ and the $x$-axis. Find the volume of the solid of revolution generated by revolving $R$ about the $x$-axis.
\begin{flushright}
(6 marks)    
\end{flushright}
\end{problem}
\newpage
\begin{problem}{2020Q10}
\newline
(a) Using integration by substitution, prove that $\int_{\frac{\pi}{12}}^{\frac{\pi}{6}} \ln \left(\sin \left(\frac{\pi}{4}-x\right)\right) \mathrm{d} x=\int_{\frac{\pi}{12}}^{\frac{\pi}{6}} \ln (\sin x) \mathrm{d} x$.\\
(b) Using (a), evaluate $\int_{\frac{\pi}{12}}^{\frac{\pi}{6}} \ln (\cot x-1) \mathrm{d} x$.
\begin{flushright}
(3 marks)    
\end{flushright}
(c) (i) Using $\cot (A-B)=\frac{\cot A \cot B+1}{\cot B-\cot A}$, or otherwise, prove that $\cot \frac{\pi}{12}=2+\sqrt{3}$.
(ii) Using integration by parts, prove that $\int_{\frac{\pi}{12}}^{\frac{\pi}{6}} \frac{x \csc ^2 x}{\cot x-1} \mathrm{~d} x=\frac{\pi}{8} \ln (2+\sqrt{3})$.
\begin{flushright}
(7 marks)    
\end{flushright}
\end{problem}
\newpage
continue 2020Q10
\newpage
\begin{problem}{2021Q4}
\newline
(a) Prove that $\cos 2 x+\cos 4 x+\cos 6 x=4 \cos x \cos 2 x \cos 3 x-1$.\\
(b) Solve the equation $\cos 4 \theta+\cos 8 \theta+\cos 12 \theta=-1$, where $0 \leq \theta \leq \frac{\pi}{2}$.
\begin{flushright}
(6 marks)    
\end{flushright}
\end{problem}
\newpage
\begin{problem}{2021Q9}
\newline
(a) Let $\frac{-\pi}{2}<\theta<\frac{\pi}{2}$.\\
(i) Find $\frac{\mathrm{d}}{\mathrm{d} \theta} \ln (\sec \theta+\tan \theta)$.\\
(ii) Using the result of (a)(i), find $\int \sec \theta d \theta$. Hence, find $\int \sec ^3 \theta d \theta$.
\begin{flushright}
(4 marks)    
\end{flushright}
(b) Let $\mathrm{g}(x)$ and $\mathrm{h}(x)$ be continuous functions defined on $\mathbf{R}$ such that $\mathrm{g}(x)+\mathrm{g}(-x)=1$ and $\mathrm{h}(x)=\mathrm{h}(-x)$ for all $x \in \mathbf{R}$.\\
Using integration by substitution, prove that $\int_{-a}^a \mathrm{~g}(x) \mathrm{h}(x) \mathrm{d} x=\int_0^a \mathrm{~h}(x) \mathrm{d} x$ for any $a \in \mathbf{R}$.
\begin{flushright}
(3 marks)    
\end{flushright}
(c) Evaluate $\int_{-1}^1 \frac{3^x x^2}{\left(3^x+3^{-x}\right) \sqrt{x^2+1}} \mathrm{~d} x$.
\begin{flushright}
(5 marks)    
\end{flushright}
\end{problem}
\newpage
continue 2021Q9
\newpage

\begin{problem}{2021Q11}
\newline
Define $P=\left(\begin{array}{cc}\sin \theta & \cos \theta \\ -\cos \theta & \sin \theta\end{array}\right)$, where $\frac{\pi}{2}<\theta<\pi$.\\
(a) Let $A=\left(\begin{array}{cc}\alpha & \beta \\ \beta & -\alpha\end{array}\right)$, where $\alpha, \beta \in \mathbf{R}$.\\

Prove that $P A P^{-1}=\left(\begin{array}{cc}-\alpha \cos 2 \theta+\beta \sin 2 \theta & -\beta \cos 2 \theta-\alpha \sin 2 \theta \\ -\beta \cos 2 \theta-\alpha \sin 2 \theta & \alpha \cos 2 \theta-\beta \sin 2 \theta\end{array}\right)$.
\begin{flushright}
(3 marks)    
\end{flushright}
(b) Let $B=\left(\begin{array}{cc}1 & \sqrt{3} \\ \sqrt{3} & -1\end{array}\right)$.\\
(i) Find $\theta$ such that $P B P^{-1}=\left(\begin{array}{cc}\lambda & 0 \\ 0 & \mu\end{array}\right)$, where $\lambda, \mu \in \mathbf{R}$.\\
(ii) Using the result of (b)(i), prove that $B^n=2^{n-2}\left(\begin{array}{cc}(-1)^n+3 & \sqrt{3}(-1)^{n+1}+\sqrt{3} \\ \sqrt{3}(-1)^{n+1}+\sqrt{3} & 3(-1)^n+1\end{array}\right)$ for any positive integer $n$.\\
(iii) Evaluate $\left(B^{-1}\right)^{555}$.
\begin{flushright}
(9 marks)    
\end{flushright}
\end{problem}
\newpage
continue 2021Q11
\newpage
\begin{problem}
Let $\frac{\pi}{4}<\theta<\frac{\pi}{2}$.\\
(a) Prove that $\frac{\tan \theta}{1-\cot \theta}+\frac{\cot \theta}{1-\tan \theta}=1+\sec \theta \csc \theta$.
(b) Solve the equation $\frac{\tan \theta}{1-\cot \theta}+\frac{\cot \theta}{1-\tan \theta}=5$.
\begin{flushright}
(5 marks)    
\end{flushright}
\end{problem}
\newpage
\begin{problem}{2022Q3}
\newline
Let $\frac{\pi}{4}<\theta<\frac{\pi}{2}$.\\
(a) Prove that $\frac{\tan \theta}{1-\cot \theta}+\frac{\cot \theta}{1-\tan \theta}=1+\sec \theta \csc \theta$.\\
(b) Solve the equation $\frac{\tan \theta}{1-\cot \theta}+\frac{\cot \theta}{1-\tan \theta}=5$.
\begin{flushright}
(5 marks)    
\end{flushright}
\end{problem}
\newpage

\begin{problem}{2022Q10}
\newline
Let $\mathrm{g}(x)=\cos ^2 x \cos 2 x$.\\
(a) Prove that $\int \mathrm{g}(x) \mathrm{d} x=\frac{\sin 2 x \cos ^2 x}{2}+\frac{1}{2} \int \sin ^2 2 x \mathrm{~d} x$.
\begin{flushright}
(2 marks)    
\end{flushright}
(b) Evaluate $\int_0^\pi \mathrm{g}(x) \mathrm{d} x$.
\begin{flushright}
(2 marks)    
\end{flushright}
(c) Using integration by substitution, evaluate $\int_0^\pi x \mathrm{~g}(x) \mathrm{d} x$.
\begin{flushright}
(4 marks)    
\end{flushright}
(d) Evaluate $\int_{-\pi}^{2 \pi} x \mathrm{~g}(x) \mathrm{d} x$.
\begin{flushright}
(4 marks)    
\end{flushright}
\end{problem}
\newpage
continue 2022Q10
\newpage
\begin{problem}{2022Q11}
\newline
(a) Let $n$ be a positive integer. Denote the $2 \times 2$ identity matrix by $I$.\\
(i) Let $A$ be a $2 \times 2$ matrix. Simplify $(I-A)\left(I+A+A^2+\cdots+A^n\right)$.\\
(ii) Let $A=\left(\begin{array}{cc}\cos \theta & -\sin \theta \\ \sin \theta & \cos \theta\end{array}\right)$, where $\theta$ is not a multiple of $2 \pi$.\\

It is given that $A^n=\left(\begin{array}{cc}\cos n \theta & -\sin n \theta \\ \sin n \theta & \cos n \theta\end{array}\right)$.\\
(1) Prove that $(I-A)^{-1}=\frac{1}{2 \sin \frac{\theta}{2}}\left(\begin{array}{cc}\sin \frac{\theta}{2} & -\cos \frac{\theta}{2} \\ \cos \frac{\theta}{2} & \sin \frac{\theta}{2}\end{array}\right)$.\\
(2) Using the result of (a)(i) and (a)(ii)(1), prove that $I+A+A^2+\cdots+A^n=\frac{\sin \frac{(n+1) \theta}{2}}{\sin \frac{\theta}{2}}\left(\begin{array}{cc}\cos \frac{n \theta}{2} & -\sin \frac{n \theta}{2} \\ \sin \frac{n \theta}{2} & \cos \frac{n \theta}{2}\end{array}\right)$.
\begin{flushright}
(7 marks) 
\end{flushright}
\newpage
(b) Using (a)(ii), evaluate\\
(i) $\quad \cos \frac{5 \pi}{18}+\cos \frac{5 \pi}{9}+\cos \frac{5 \pi}{6}+\cdots+\cos 25 \pi$;\\
(ii) $\quad \cos ^2 \frac{\pi}{7}+\cos ^2 \frac{2 \pi}{7}+\cos ^2 \frac{3 \pi}{7}+\cdots+\cos ^2 7 \pi$.
\begin{flushright}
(6 marks)    
\end{flushright}
\end{problem}
\newpage
continue 2022Q11
\newpage
\begin{problem}
Let $\mathrm{f}(x)=-x \sin x$.\\
(a) Prove that $\mathrm{f}\left(\frac{\pi}{2}+h\right)-\mathrm{f}\left(\frac{\pi}{2}\right)=\pi \sin ^2\left(\frac{h}{2}\right)-h \cos h$.\\
(b) Using (a), find $\mathrm{f}^{\prime}\left(\frac{\pi}{2}\right)$ from first principles.
\begin{flushright}
(5 marks)    
\end{flushright}
\end{problem}
\newpage
\begin{problem}{2023Q3}
\newline
(a) Find a pair of constants $p$ and $q$ such that $11 \sin x+7 \cos x \equiv p(3 \sin x+\cos x)+q(3 \cos x-\sin x)$.\\
(b) Evaluate $\int_0^{\frac{\pi}{4}} \frac{11 \sin x+7 \cos x}{3 \sin x+\cos x} \mathrm{~d} x$.
\begin{flushright}
(6 marks)    
\end{flushright}
\end{problem}
\newpage
\begin{problem}{2023Q4}
\newline
(a) Prove that $\cos 3 x=4 \cos ^3 x-3 \cos x$.\\
(b) Using (a), solve the equation $\sec ^3 x-6 \sec ^2 x+8=0$, where $\frac{\pi}{2}<x<\frac{3 \pi}{2}$.
\begin{flushright}
(5 marks)  
\end{flushright}
\end{problem}
\newpage
\begin{problem}{2023Q8}
\newline
(a) Let $\theta \in \mathbf{R}$. Using mathematical induction, prove that $\sin \theta \sum_{k=1}^n \sin 2 k \theta=\sin n \theta \sin (n+1) \theta$ for all positive integers $n$.\\
(b) Using (a), find a pair of rational numbers $a$ and $b$ such that $\sum_{k=1}^{111} \sin \frac{k \pi}{11} \cos \frac{k \pi}{11}=a \sin b \pi$, where $0<b<\frac{1}{2}$.
\begin{flushright}
(8 marks)    
\end{flushright}
\end{problem}
\newpage
\begin{problem}{2023Q12}
\newline
(a) Let $a$ be a non-zero constant. Prove that $\int_0^1 x^2 e^{a x} \mathrm{~d} x=\frac{\left(a^2-2 a+2\right) e^a-2}{a^3}$.
\begin{flushright}
(3 marks)    
\end{flushright}
(b) Using (a) and integration by substitution, evaluate $\int_0^{e-1} x(\ln (1+x))^2 \mathrm{~d} x$.
\begin{flushright}
(4 marks)    
\end{flushright}
(c) Evaluate $\int_0^{\frac{\pi}{2}}(\ln (1+(e-1) \cos x))^2 \sin 2 x \mathrm{~d} x$.
\begin{flushright}
(3 marks)    
\end{flushright}
(d) Evaluate $\int_{\frac{\pi}{2}}^\pi(\ln (1+(e-1) \sin x))^2 \sin 2 x \mathrm{~d} x$.
\begin{flushright}
(3 marks)    
\end{flushright}
\end{problem}
\newpage
\begin{problem}{2024Q4}
\newline
(a) Let $0<x<\frac{\pi}{2}$. Prove that $\csc 2 x-\cot 2 x=\tan x$.\\
(b) Solve the equation $(\csc 3 \theta-\cot 3 \theta)(\csc \theta-\cot \theta)=1$, where $\frac{\pi}{6}<\theta<\frac{\pi}{3}$.
\begin{flushright}
(5 marks)    
\end{flushright}
\end{problem}
\newpage
\begin{problem}{2024Q5}
\newline
(a) Let $k$ be a constant.\\
Using integration by parts, prove that $\int \cos (k \ln x) \mathrm{d} x=\frac{x}{1+k^2}(\cos (k \ln x)+k \sin (k \ln x))+$ constant .\\
(b) Using (a), or otherwise, evaluate $\int_1^e \sin ^2(\pi \ln x) \mathrm{d} x$.
\begin{flushright}
(6 marks)    
\end{flushright}
\end{problem}
\newpage
\begin{problem}{2024Q6}
\newline
Consider the system of linear equations in real variables $x, y, z$

$$
(E):\left\{\begin{array}{l}
3 x+y-9 z=0 \\
2 x+y-7 z=0
\end{array}\right.
$$

(a) Solve $(E)$.\\
(b) Someone claims that ( $E$ ) has a unique solution ( $x, y, z$ ) satisfying $\sin x+\cos y-\cos z=0$, where $0<z<\frac{\pi}{2}$. Do you agree? Explain your answer.
\begin{flushright}
(6 marks)    
\end{flushright}
\end{problem}
\newpage
\begin{problem}{2024Q9}
\newline
Two toy cars, $A$ and $B$, move along a straight line towards east with constant speeds 4 metres per second and 1 metre per second respectively. They start at the point $O$ at the same time. The fixed point $C$ is 10 metres due north of $O$. The following figure shows the positions of $A$ and $B$ after $t$ seconds. Let $\angle A C B=\theta$ radians .\\
\begin{figure}[h]
    \centering
    \includegraphics[width=0.5\linewidth]{2024Q9.png}
    \caption{Caption}
    \label{fig:placeholder}
\end{figure}
\newline
(a) Prove that $\tan \theta=\frac{15 t}{2\left(t^2+25\right)}$.\\
(b) It is given that $\angle B A C=\angle A C B$ when $t=T$. Find\\
(i) $T$,\\
(ii) the rate of change of $\theta$ when $t=T$.
\begin{flushright}
(7 marks)    
\end{flushright}
\end{problem}
\newpage
continue 2024Q9
\newpage
\begin{problem}{2024Q11}
\newline
(a) Let $a$ be a non-zero constant. Using integration by substitution, find $\int \frac{1}{x^2+a^2} \mathrm{~d} x$ in terms of $a$.
\begin{flushright}
(3 marks)    
\end{flushright}
(b) Let $\mathrm{g}(x)$ and $\mathrm{h}(x)$ be continuous functions defined on $\mathbf{R}$. It is given that $\mathrm{g}(x)$ is an even function and $\mathrm{h}(x)$ is an odd function. Prove that $\int_{-c}^c \frac{\mathrm{~g}(x)}{1+e^{\mathrm{h}(x)}} \mathrm{d} x=\int_0^c \mathrm{~g}(x) \mathrm{d} x$, where $c$ is a constant.
\begin{flushright}
(4 marks)    
\end{flushright}
(c) Evaluate $\int_{-1}^1 \frac{3^x+3^{-x}}{\left(1+e^{\sin ^3 x}\right)\left(9^x+9^{-x}+7\right)} \mathrm{d} x$.
\begin{flushright}
(5 marks)    
\end{flushright}
\end{problem}
\newpage
continue 2024Q11
\newpage
\begin{problem}{2025Q2}
\newline
Let $\mathrm{f}(x)=\frac{2 x}{\tan x}$, where $0<x<\frac{\pi}{2}$.\\
(a) Prove that $\mathrm{f}\left(\frac{\pi}{4}+h\right)-\mathrm{f}\left(\frac{\pi}{4}\right)=\frac{2 h-2 h \tan h-\pi \tan h}{1+\tan h}$.\\
(b) Find $\mathrm{f}^{\prime}\left(\frac{\pi}{4}\right)$ from first principles.
\begin{flushright}
(5 marks)    
\end{flushright}
\end{problem}
\newpage
\begin{problem}{2025Q6}
\newline
Let $\frac{\pi}{2}<x<\frac{3 \pi}{4}$.\\
(a) Find a pair of constants $h$ and $k$ such that $\frac{\sec x+\tan x}{\csc x-\cot x}+\frac{\sec x-\tan x}{\csc x+\cot x} \equiv h \sec x \csc x+k$.\\
(b) Solve the equation $\frac{\sec x+\tan x}{\csc x-\cot x}+\frac{\sec x-\tan x}{\csc x+\cot x}+6=0$.
\begin{flushright}
(5 marks)    
\end{flushright}
\end{problem}
\newpage
\begin{problem}{2025Q9}
\newline
(a) Using integration by substitution, prove that $\int \frac{1}{\left(1+x^2\right)^2} \mathrm{~d} x=\frac{1}{2} \tan ^{-1} x+\frac{x}{2\left(1+x^2\right)}+$ constant.\\
(b) Consider the curve $C: y=\frac{1}{1+3 x^2}$. Let $R$ be the region bounded by $C$, the straight line $x=1$ and the two axes. Find the volume of the solid of revolution generated by revolving $R$ about the $x$-axis.
\begin{flushright}
(8 marks)    
\end{flushright}
\end{problem}
\newpage
\begin{problem}{2025Q11}
\newline
(a) Let $m$ be a positive integer. Express $\int_0^\pi e^{-x} \sin 2 m x \mathrm{~d} x$ in terms of $m$.
\begin{flushright}
(3 marks)    
\end{flushright}
(b) (i) Evaluate $\int_0^\pi e^{-x} \sin 5 x \cos 3 x d x$.\\
(ii) Evaluate $\int_0^{2 \pi} e^{-x} \sin 5 x \cos 3 x \mathrm{~d} x$.
\begin{flushright}
(6 marks)    
\end{flushright}
(c) Evaluate $\int_{e^{\frac{-3 \pi}{2}}}^{e^{\frac{\pi}{2}}} \cos \left(\ln x^5\right) \sin \left(\ln x^3\right) \mathrm{d} x$.
\begin{flushright}
(4 marks)    
\end{flushright}
\end{problem}
\newpage
continue 2025Q11
\newpage

\end{document}
